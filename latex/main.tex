\documentclass[10pt]{article}
\usepackage[utf8]{inputenc}
\usepackage[T1]{fontenc}
\usepackage{amsmath,amssymb}
\usepackage{geometry}
\begin{document}

\section{Préparation}

\subsection{Calcul du gradient \(\nabla f(x,y)\)}

La fonction de Rosenbrock considérée est
\[
    f(x,y) \;=\; (1 - x)^2 \;+\; a^2\,(x - y^2)^2.
\]
Pour tout \((x,y)\in\mathbb{R}^2\), on calcule les dérivées partielles :
\[
\frac{\partial f}{\partial x}(x,y) 
= -2\,(1 - x) \;+\; 2\,a^2\,(x - y^2),
\qquad
\frac{\partial f}{\partial y}(x,y) 
= -4\,a^2\,y\,(x - y^2).
\]
Le gradient est donc
\[
    \nabla f(x,y) 
    = 
    \begin{pmatrix}
    -2\,(1 - x) + 2\,a^2\,(x - y^2) \\[6pt]
    -4\,a^2\,y\,(x - y^2)
    \end{pmatrix}.
\]

\subsection{Points critiques : résolution de \(\nabla f(x,y) = 0\)}

Le système \(\nabla f(x,y) = 0\) s’écrit :
\[
    \begin{cases}
      -2\,(1 - x) + 2\,a^2\,(x - y^2) = 0,\\[4pt]
      -4\,a^2\,y\,(x - y^2) = 0.
    \end{cases}
\]
\begin{itemize}
\item \textbf{Cas 1 : \(y = 0\).}  
  De la première équation, on obtient :
  \[
    -2\,(1 - x) + 2\,a^2\,x = 0 
    \quad\Longrightarrow\quad
    x\,(1 + a^2) = 1
    \quad\Longrightarrow\quad
    x = \frac{1}{1 + a^2}.
  \]
  D’où le point critique
  \(\displaystyle
    a_1 = \Bigl(\tfrac{1}{1 + a^2},\,0\Bigr).
  \)

\item \textbf{Cas 2 : \(x - y^2 = 0\).}  
  Ici, \(x = y^2\). On remplace dans la première équation :
  \[
    -2\,(1 - y^2) + 2\,a^2\,(y^2 - y^2) 
    = -2\,(1 - y^2) = 0
    \quad\Longrightarrow\quad
    y^2 = 1.
  \]
  Donc \(y = \pm 1\) et \(x = 1\). On obtient deux autres points critiques :
  \(\displaystyle
    a_2 = (1,\,1), 
    \quad
    a_3 = (1,\,-1).
  \)
\end{itemize}

\subsection{Matrice Hessienne \(\mathrm{Hess}_f(x,y)\)}

En dérivant une seconde fois :
\[
    \nabla f(x,y) 
    = 
    \begin{pmatrix}
    -2\,(1 - x) + 2\,a^2\,(x - y^2) \\
    -4\,a^2\,y\,(x - y^2)
    \end{pmatrix}.
\]
on trouve :
\[
  \frac{\partial^2 f}{\partial x^2} 
  = 2\,(1 + a^2),
  \qquad
  \frac{\partial^2 f}{\partial x \partial y} 
  = \frac{\partial^2 f}{\partial y \partial x} 
  = -4\,a^2\,y,
  \qquad
  \frac{\partial^2 f}{\partial y^2} 
  = -4\,a^2\,x + 12\,a^2\,y^2.
\]
Ainsi,
\[
\mathrm{Hess}_f(x,y)
  = 
  \begin{pmatrix}
    2\,(1 + a^2) & -4\,a^2\,y \\[6pt]
    -4\,a^2\,y   & -4\,a^2\,x + 12\,a^2\,y^2
  \end{pmatrix}.
\]

\subsection{Nature des points critiques \((a_1,a_2,a_3)\)}

Pour chaque point, on évalue \(\mathrm{Hess}_f\) :

\begin{itemize}
\item
  \(\displaystyle a_1 
    = \Bigl(\tfrac{1}{1 + a^2},\,0\Bigr)\).
  La hessienne devient alors diagonale avec 
  \(2\,(1 + a^2)\) et \(\displaystyle -4\,a^2\,\Bigl(\tfrac{1}{1 + a^2}\Bigr)\) 
  sur la diagonale, donc l’une est positive et l’autre négative.
  \(\mathrm{Hess}_f(a_1)\) est par conséquent indéfinie :
  \(a_1\) est un \textbf{point selle}.
\item
  \(\displaystyle a_2 = (1,\,1)\).
  Dans ce cas,
  \[
    \mathrm{Hess}_f(a_2)
    = 
    \begin{pmatrix}
      2\,(1 + a^2) & -4\,a^2 \\
      -4\,a^2      & 8\,a^2
    \end{pmatrix}.
  \]
  Le déterminant est strictement positif et la trace est positive ; les valeurs propres sont toutes positives :
  \(a_2\) est donc un \textbf{minimum local}.
\item
  \(\displaystyle a_3 = (1,\,-1)\).
  On obtient de même
  \[
    \mathrm{Hess}_f(a_3)
    = 
    \begin{pmatrix}
      2\,(1 + a^2) & 4\,a^2 \\
      4\,a^2       & 8\,a^2
    \end{pmatrix},
  \]
  dont le déterminant et la trace sont aussi positifs : on a deux valeurs propres positives ;
  \(a_3\) est un \textbf{minimum local}.
\end{itemize}

Par ailleurs, on vérifie que \(f(a_2) = f(a_3) = 0\), correspondant à la valeur minimale (globale) de la fonction.

\subsubsection*{Conclusion}

On a donc identifié trois points critiques : 
\[
  a_1 = \Bigl(\tfrac{1}{\,1+a^2\,},\,0\Bigr)\quad(\text{point selle}),\quad
  a_2 = (1,1),\quad
  a_3 = (1,-1)\quad(\text{minima}).
\]
\end{document}